
% ================= IF YOU HAVE QUESTIONS =======================
% Questions regarding the SIGS styles, SIGS policies and
% procedures, Conferences etc. should be sent to
% Adrienne Griscti (griscti@acm.org)
%
% Technical questions _only_ to
% Gerald Murray (murray@hq.acm.org)
% ===============================================================
%
% For tracking purposes-this is V2.0-May 2012
%\documentclass[letterpaper, twocolumn, 10pt]{article}
\documentclass{sigcomm-alternate}
\usepackage{setspace}
\usepackage{cite}
\usepackage{float}
\begin{document}
%
% --- Author Metadata here ---
%\CopyrightYear{2007} % Allows default copyright year (20XX) to be over-ridden-IF NEED BE.
%\crdata{0-12345-67-8/90/01}  % Allows default copyright data (0-89791-88-6/97/05) to be over-ridden-IF NEED BE.
% --- End of Author Metadata ---
\title{An Exploration of the Connectivity, Technology, and Topology of the Internet Worldwide} 
%
% You need the command \numberofauthors to handle the 'placement
% and alignment' of the authors beneath the title.
%
% For aesthetic reasons, we recommend 'three authors at a time'
% i.e. three 'name/affiliation blocks' be placed beneath the title.
%
% NOTE: You are NOT restricted in how many 'rows' of
% "name/affiliations" may appear. We just ask that you restrict
% the number of 'columns' to three.
%
% Because of the available 'opening page real-estate'
% we ask you to refrain from putting more than six authors
% (two rows with three columns) beneath the article title.
% More than six makes the first-page appear very cluttered indeed.
%
% Use the \alignauthor commands to handle the names
% and affiliations for an 'aesthetic maximum' of six authors.
% Add names, affiliations, addresses for
% the seventh etc. author(s) as the argument for the
% \additionalauthors command.
% These 'additional authors' will be output/set for you
% without further effort on your part as the last section in
% the body of your article BEFORE References or any Appendices.
%\numberofauthors{4} %  in this sample file, there are a *total*
% of EIGHT authors. SIX appear on the 'first-page' (for formatting
% reasons) and the remaining two appear in the \additionalauthors section.
%
\numberofauthors{3}
\author{
% You can go ahead and credit any number of authors here,
% e.g. one 'row of three' or two rows (consisting of one row of three
% and a second row of one, two or three).
%
% The command \alignauthor (no curly braces needed) should
% precede each author name, affiliation/snail-mail address and
% e-mail address. Additionally, tag each line of
% affiliation/address with \affaddr, and tag the
% e-mail address with \email.
%
% 1st. author
\alignauthor
{Thomas Kennedy}\\
       \affaddr{Southern Methodist University}\\
       \email{tkennedy@smu.edu}\\
% 2nd. author
\alignauthor
{Alec Siems}\\
       \affaddr{Southern Methodist University}\\
       \email{asiems@smu.edu}\\
\and  % use '\and' if you need 'another row' of author names
% 3rd. author
\alignauthor 
{Vipul Kohli}\\
       \affaddr{Southern Methodist University}\\
       \email{vkohli@smu.edu}
}
% There's nothing stopping you putting the seventh, eighth, etc.
% author on the opening page (as the 'third row') but we ask,
% for aesthetic reasons that you place these 'additional authors'
% in the \additional authors block, viz.
\date{February 4 2015}
% Just remember to make sure that the TOTAL number of authors
% is the number that will appear on the first page PLUS the
% number that will appear in the \additionalauthors section.

\maketitle

\begin{abstract}
Exploring and understanding the topology of a computer network provides valuable insight into the distribution of traffic, properties of central nodes, and the hierarchical structures of the network. The purpose of this paper is to represent the connectivity and physical media used throughout the Internet with a graph-based model as well as to explore the topology of international networks. This purpose is achieved through research and experimentation into Internet Service Provider (ISP) network maps, country demographics, and software tools for tracking packets through the Internet. Additionally, this paper explores the effects that network politics and privacy to the global Internet connectivity could potentially have on the routing patterns between the different types of ISP's that service both edge and transit core networks. 

% Some effects of network politics and privacy are due to the Border Gateway Protocol (BGP), firewalls, and longer distance routing.
\end{abstract}

\section{Introduction}
The Internet is a very large and dynamic network of networks where routers and computers are constantly being added and dropped while data packets are simultaneously routed all over the world. Learning about and exploring a network’s topology can help researchers and developers create more efficient ways to route data within or through the network; typically, this would improve overall data transmission speeds within the network.

	The topology of a network is usually modeled using a graph where the nodes are computers or routers while the edges represent direct connections between these computers or routers (edge weights can be used to represent the amount of traffic or the type of transmission medium between the two nodes). A full network graph essentially shows all the paths that packets can take to get from any arbitrary source node to any arbitrary destination node within the network. Just by examining a proper network graph, a researcher can begin to discover unique properties of the network. Some of the nodes that would be most noticeable to the researcher are likely the nodes that tend to be more centralized in the network and have many more connections with other nodes on the network than average. Central nodes are nodes that have many edges connected to them; therefore, they handle much more traffic than other nodes on the network due to their centered location–where they can provide many route options to many nodes–and their capability to handle larger amounts of traffic. Additionally, edge weights can be added to the graph to help guide the reader's attention to connections that have a lot of traffic routed through them (edge weights would correspond to traffic volume) or to common/rare transmission technologies found in the network (edge weights would correspond to the different transmission technologies). This information is useful to a network engineer who would, after thoroughly examining the network graph, ensure that central nodes have the proper technology to handle their demand for traffic while high-traffic edges are kept to a minimum between nodes that aren't designed to handle as much traffic.
	
	There are numerous ways that these network graphs may be modified so that they provide more useful information.  These node weight values could help a researcher identify, for example, discrepancies within the network between a node's weight (how many packets traveled through the node during the data collection) and the node's number of edges (the number of connections between the node and other nodes in the network). Should discrepancies be found, this information would alert the researcher to the possibility that traffic is flowing inefficiently through the network. The nodes with the most edges should receive the most traffic because they would be more centralized and generally allow for more efficient routing. Additionally, the size, shape, and color of the nodes may be changed to make the graph more meaningful or easier to interpret. For example, to explore how the Internet routes packets throughout the world, it is more helpful to explore a graph with cities or states on a world map as the nodes (colored by amount of traffic) than it is to explore a plain white-and-black graph with circles representing the nodes. 
	
	These methods are suitable for basic network graphs that have a manageable amount of nodes and edges. However, the Internet is far from basic; in fact, the Internet is very complex. It is a network of networks–imagine if each node on a graph represented an entirely different graph located at that node. Internet Service Providers (ISP's) are network graphs that can be used to construct a network map of the Internet. ISP's are typically companies that own large amounts of expensive Internet technologies and are used to route Internet traffic into, within, and out of their ISP network in order to help packets reach their final destinations. These ISP's exist in many different sizes ranging from only covering households and neighborhoods to spanning entire continents and multiple countries. 
	
	At some point (unless the destination node exists within the same ISP as the source node) one ISP will have to pass a packet on to a peer ISP. The network knows where to route the packet because each ISP maintains a Border Gateway Protocol (BGP) table. ISP's consult their BGP tables when they need to route a packet outside their network. In order for this to work properly, however, each BGP table (usually maintained separately by individual routers in the ISP, but synced often to ensure consistency) must stay up-to-date with both it's own ISP and peer/neighboring ISP's by asking the network for BGP table updates. 
	
	In this article, we explore the topology of the Internet and attempt to graphically represent the connectivity and physical media used throughout the Internet. After collecting traceroute data for approximately 36,000 source/destination pairs worldwide, we generated several graphs to help us understand the topology of the Internet as well as discover the central, highly utilized ISP's and cities.
% \end{Introduction}

\section{Related Work}
Studies relating to Internet topology often differ in their methods and assumptions. This section examines research relevant to 
In order to most effectively explore the topology and connectivity of the Internet, we must first garner a solid understanding of the research already done up to this point as well as an understanding of the technologies and protocols that make the Internet work.

% The properties of load balancing and hierarchical support of these protocols are relevant. 

Tannenbaum and Wetherall \cite{tannenbaum} identify the basic architecture of the connectivity between routers and destination hosts on the Internet. They identify Open Shortest Path First (OSPF) and Intermediate-System (IS-IS) as the intradomain routing protocols in company networks and ISP networks, respectively. Every independent network or autonomous system (AS) has a backbone area which contains the routers connecting all the peripheral areas. This is why intradomain networks can stretch long distances geographically. However, Tannenbaum and Wetherall did not attempt to generate any network traversal data or graphics to help explore the topology of the Internet. 

% Their Traceroute program sent out small datagram (UDP) packets, with different hop counts set for each packet’s time-to-live (TTL) field of the header. With each hop, the TTL was decremented until finally reaching zero. When the TTL of the packet was zero, the receiving node returned an Internet Control Message Protocol (ICMP) packet to their sender reporting “TTL exceeded.” Cheswick and Burch interpreted these ICMP messages as route information corresponding to that packet. While most of the ICMP packets were successfully received by their Traceroute source, some packets failed to return due to timeouts and firewalls along the way. They were blindly sending packets to the large list of numerical addresses without any expectation of packets reaching working hosts or UDP services. However, ICMP returned information for each successful hop.

Cheswick and Burch \cite{ches} collected Traceroute data of more than 100,000 nodes on the Internet in June, 1999. They documented their Traceroute results well which helped us interpret the results from the similar Traceroute program we used for our research. In order to learn more about the IP addresses from Traceroute, they documented three different possible IP address resolutions through the Domain Name System (DNS) for each of the packet's hops. If the domain name associated with the IP address of the router is available, Traceroute shows the domain name. Resolved domain names typically contain ISP identification details which are very helpful for understanding the topology of the Internet, given that enough ISP details are available. If the domain is not configured, Traceroute clearly states so but still gives the IP address. Otherwise, only an IP address of the router is returned, meaning that it is not associated to any domain name.

% Their Traceroute program sent out small datagram (UDP) packets, with different hop counts set for each packet’s time-to-live (TTL) field of the header. With each hop, the TTL was decremented until finally reaching zero. When the TTL of the packet was zero, the receiving node returned an Internet Control Message Protocol (ICMP) packet to their sender reporting “TTL exceeded.” Cheswick and Burch’s Traceroute interpreted these ICMP error reporting packets as route information for a packet traveling to its destination.

% The Traceroute program used in this research also used tracing via ICMP response packets. However, the packets sent were TCP/IP packets bound for active servers that were hosting websites. Therefore, if the connection to the IP address was successful, a web browser on the sending host was expected to display a web page. Cheswick and Burch’s documentation of Traceroute was very useful.

% Cheswick and Burch visualized their data on several graphs of color-coded nodes and links. Their graphs served as inspiration for our research in mapping the Internet of 2015, when more meaningful metrics can be measured because the Internet has become a ubiquitous tool for people and software on an international scale. Cheswick and Burch found regional hotspots to identify network topology and connectivity in 1999. Their 1999 graphs show the different topologies of ISP networks. Similarly, our research focuses on identifying hotspots in the Internet of 2015, which has many more ISP's and network nodes to sample compared to 1999. However, the Internet can't scale much more before it runs into issues.

% From May 1 to May 10, 1999, Cheswick and Burch monitored the reach of the Internet in Yugoslavia during the Serbian war using Traceroute. Nodes were disappearing and reappearing daily as sections of the power grid were lost and restored in Serbia and Bosnia throughout the war. The erection of firewalls and privacy on entire networks has a similar effect on connectivity, in 2015, as we observe in our research.

The Internet Architecture Board (IAB) not only has identified an Internet scalability problem with the current address space (IPv4), but also has identified significant disadvantages with switching from IPv4 to IPv6. The two solutions that the IAB gives to alleviate this problem are: separation of edge networks from the transit core or elimination through multihoming the edge networks \cite{scalability}. The separation solution must involve adding a layer between the edge network and the transit core to handle control and management while the elimination solution requires edge networks to accept addresses from their ISP's. This is a great example of how clever network design can offer many benefits, including saving address space on the IPv4 addressing protocol.  

Many ISP's include a map of their network on their website that shows their data centers and the types of physical media used to connect them. Some studies have generated maps showing network or population statistics while other studies have taken different approaches to map all the connections on the Internet from a particular point of view. A particular study used ICMP ping, nmap port scans, reverse DNS records, and Traceroute records to determine that over 3.7 billion hosts exist on the Internet as of December, 2012. However, authenticity of the data in this study was a serious concern \cite{botnet}. Just as Krence, Hohlfeld, and Feldmann criticized and thoroughly examined the quality of the network data they collected in the aforementioned study, we held the authenticity of our data collected as a critical factor for inferences drawn from the data. In order to explain some anomalies in our data that , we cross-checked across multiple sources to ensure that every step was done correctly. 

% This study also emphasized the interest of the general public in capturing the size of the Internet. This public interest appreciates the value of the field of research of routing across the worldwide Internet.

% Spring, Mahajan, and Wetherall \cite{topology} from the University of Washington addressed the task of mapping ISP's in their article, “Measuring ISP Topologies with Rocketfuel." The goal of their work was to obtain realistic and reliable data regarding the router level distribution of ISP networks. The expected end result would be an ISP map consisting of all sources, destinations, and routing servers composing the multiple Autonomous Systems (AS's) that make up the ISP's network. After collecting all intermediary hops from sending packets through the network, they categorized each hop as either a Backbone, Core, or Access router based on particular properties of the router. This resulted in several nice looking network graphs, each representing an individual ISP's network, over a map of the United States with connections between various cities on the network represented as lines.  				
					
% Andersen, Feamster, Bauer, and Balakrishna \cite{inference} of MIT addressed similar issues in their article, “Topology Inference from BGP Routing Dynamics." Like the researchers from the University of Washington, the MIT researchers also used Rocketfuel to route packets across the country. However, instead of a geographic representation of this data similar to what the researchers at the University of Washington generated, the MIT group gathered measurements of speed and reliability on each ISP when accessing routes via BGP tables. The unique part of this research was that the researchers inferred the network's topology based on BGP table updates and messages then mapped nodes based on their common ISP and AS instead of just flooding packets into the network via Traceroute to collect network data. 


\section{Research Methodology}
This sections describes the tools and procedure used to collect and analyze ISP and routing data in order to produce a visualization of the Internet to every country and continent which shows the topology, material, and routing involved.

\subsection{Source and Destination Selection}
In order to map the worldwide Internet, a list of IP addresses of servers in each country in the world was required. Therefore, we chose a list of URLs of the Web sites of universities in every country provided by Univ.cc [3]. We targeted university Web sites over commercial Web sites since universities are more likely to have their Web servers in-house whereas commercial Web sites mostly outsource their server needs. Each country’s list from Univ.cc is proportional to the country’s area.

After the URLs were collected, the next task was to determine the Internet routing to each of the university URLs. The IP2Location Traceroute [4] software converts these URLs to their respective IP addresses and determines the route to each of the IP addresses from the IP address of the chosen IP2Location host server. The IP2Location Traceroute application allows users to choose a server either in the United States (Phoenix, Dallas, or Los Angeles), Canada (Montreal), Israel (Tel Aviv), France (Paris), Germany (Dusseldorf), Netherlands (Amsterdam), England (London), Singapore, Japan (Tokyo), or Malaysia (Kuala Lumpur), a total of twelve servers in ten countries on three continents. The source server IP addresses are available, and these IP addresses would become the destinations for our bidirectional route trace attempts using worldwide VPN software tools available to us.

The Center for Applied Internet Data Analysis (CAIDA) ranks several global ISPs along with their AS numbers. These AS numbers were applied in our experiment to Daniel Austin’s BGP Lookup Tool to achieve an IP prefix which was then supplied to the BGPlay tool. CAIDA’s ranked table of ISPs and their AS numbers provided the starting nodes for our BGP research. The top AS numbers were converted to IP prefixes through Austin’s BGP lookup tool which are the parameters used in BGP routing.

\subsection{Software Tools}
IP2Location provided each node’s city, state, and country of physical location. We authenticated IP2Location's database as containing physical locations by checking controversial offshore island URLs and Antarctica IP addresses both on the IP2Location and the RIPE databases. Some software packages misinterpret a company headquarter location as the server's physical location or incorrectly approximates the location simply by the IP address.  

With accurate worldwide location information available to us, our aim became to originate the Traceroute from a few countries but reach all the countries with plans to trace the networks of all the major global ISPs and their hotspot cities.

IP2Location's Traceroute was an effective tool is routing to each of the countries, but we also wanted to generalize the behavior of routing in the reverse direction from the destination countries to the IP2Location server locations. To do so, we used VisualRoute and HMA Pro VPN. Virtual Private Networks (VPNs) allow remote hosts to connect to resources as if being local on a foreign local area network (LAN). With the HMA Pro VPN, we were able to access the Internet and run Traceroute from the point of view of a server in any country of our choice with the IP address available to verify the location in IP2Location. VisualRoute is a Traceroute geographical mapping tool that originates based on our laptops' Internet connection which would be in another country with the use of the HMA Pro VPN.

VisualRoute was more effective than IP2Location in identifying firewalls and unresponsiveness en route. However, IP2Location was more knowledgeable in conversion from router IP address to physical location on successful hops. VisualRoute attempted to estimate the location of IP addresses unknown to its database, and IP addresses showing unknown locations and ISPs were common in VisualRoute that were identifiable with IP2Location. Bidirectional generalizations were possible in both the doubly-connected cases and the singly-connected cases where firewalls and privacy were involved.

BGPlay is a JavaScript visualization tool that generates a graph representation of the links within five hops from the source IP prefix. Each of the linking nodes consist of the AS and ISP registration details which were investigated further to determine interdomain connectivity.

These tools allowed us to determine the ISP connections to all the countries of the world. With the ISP data, we were identify the physical media through research into ISP network maps and readily available lists of mobile carriers in each of the different countries.
 

\subsection{Measurement Methodology}
Using IP2Location, we ran all 36,000 traceroutes from the predetermined sources to each international university. Using a customized Java program, we collected every table of Traceroute information, included in that is Hop number, IP address, Host name, Geographic location, and the time it took to make that hop (all in the range of milliseconds).  This returned an HTML file holding hundreds of thousands of lines of information, all that needed to be parsed and aggregated into something that can be visually represented. To parse this data, a custom Python file was written for every metric we wanted to find, taking into account the Host name and geographic location. For the scope of this project, we did not identify unique IP addresses or transmission time. Using Python we counted a number of metrics, including individual city hits, to country hits, to ISP information inferred from the Host name. Each individual line of data was useless, but when pulling everything together we saw definitive patterns. Major cities like Seattle and Paris stuck out, or ISPs like Cogent and AT\&T. We then took the parsed data and piped it to a Google Charts Javascript file, which drew our data out on maps and bar charts when needed.

%\vfill
%\columnbreak

\section{Result Analysis}
This section builds upon the methodology in Section 3 by exploring the GeoCharts generated based on the Traceroute data and generalizing on our discoveries of bidirectional routing, BGP, and the physical media.
 
\subsection{Source Server Distribution}
	Another flaw in our methodology is the distribution of our source servers for the traceroute call. We used 12 sources from 10 countries around the world. As expected, there is a disproportionate number of hits in countries that have these source servers. The United States holds three of these sources in Dallas, Los Angeles, and Phoenix. Below, in Figure 1, you’ll see the map representing the number of hits in each country divided by the total population of that country with stars representing source server locations. Each of the source countries is bright green as expected, however, there are also a few countries that still had high traffic per capita despite not holding a source server. Green countries like Australia, Sweden and Iceland all fall into the top 10 countries in the world in GDP per citizen. GDP per citizen is a good metric to estimate the country’s Internet infrastructure because developed, wealthier countries invest more in Internet and networking technologies than undeveloped, poorer countries. Additionally, GDP per citizen can be used to estimate the number of universities in that country because developed, wealthier countries invest more in universities than undeveloped, poorer countries. Oman, on the other hand, is not seen at the top of the GDP per citizen list. The home of only 11 universities, Oman is the outlier as far as traceroute traffic. So while there are countries that stick out, we consistently see that the countries holding one of the source servers will have the most traffic according to the traceroute data and countries with few universities will not have much traffic. 

	
\begin{figure}[h!]
  \caption{Hits Per Country Per Capita}
  \centering
    \includegraphics[width=0.5\textwidth]{HitsPerCapitaStars.png}
\end{figure}

\subsection{Remote Server Location}

	Another flaw in our methodology would be our lack of consideration for university servers hosted in a country outside of the country where the university is located. Below you’ll find a table from our traceroute data that displays the source, destination, and every hop from the IP2Location server (Dallas, in this case) to afghanuniversity.edu.af (Figure 2). It’s easy to see that this university is native to Afghanistan, and we would expect the server to be hosted in Afghanistan. However, this Afghanistan university’s server is actually hosted in Chicago, Illinois and the traceroute shows that the packets never leaves the United States. This could drastically change our results. We think we are sending packets from the United States to Afghanistan, however, these packets never reach Afghanistan and find their destinations in Chicago, Illinois, where their servers are located, instead. It is possible that this issue had a considerable effect on our data and unfairly skews the data towards countries that host multiple universities. These countries could be selected for any variety of reasons to remotely host servers including better Internet infrastructure, cheaper server space, and/or location where the university expects the most traffic from.
	
\begin{figure}[h!]
  \caption{Dallas to afghanuniversity.edu.af}
  \centering
    \includegraphics[width=0.5\textwidth]{AfghanTraceroute2.png}
\end{figure}	

\subsection{City Traffic}
When discovering the North American cities with the most traffic, the cities we discovered made sense considering how ISPs are mapped across the globe. Peering points for multiple ISPs are located within each of the North American cities with the most traffic. ISPs use these peering points to pass packets between each other according to the BGP tables of the individual ISPs. In Figure 3, the North American cities with the most traffic are identified. As we expected, many of them are members of Tier 1 peering exchanges. The BGP tables route traffic to these cities to swap over to different ISPs, and then continue on to it’s intended destination. In the data aggregated for this map, we’ll look at three major US cities: Seattle, Dallas and Phoenix. Seattle led all cities by far, with over 21,000 hits, with Dallas around 14,000 hits, and Phoenix at approximately 7,000 hits. 

Of the three cities, Dallas and Phoenix were cities that hosted source servers for IP2Location. Therefore, the mere 7,000 hits in Phoenix is representative of the fact that a majority of the traffic starts in Phoenix while not much other traffic is routed through the city. Phoenix can be described as a high source traffic, low transmission traffic city. 

The second city we will inspect is Dallas, Texas. With approximately 14,000 hits, the city has twice as much traffic as Phoenix, yet still very short of Seattle. Dallas is a good example of a city with a good spread of traffic because it was a source node and still has a fair amount of transmission traffic flow through it. Of the approximate 3,000 universities resulting in 3,000 traceroutes from each source, Dallas has about 11,000 hits greater than the minimum amount of hits expected for Dallas. This means that many other nodes route through Dallas, which is reasonable given Dallas is a peering hub for multiple Tier 1 ISPs. Dallas can be described as a high source traffic, high transmission traffic city. 

The third city we will inspect is Seattle, Washington. Seattle is a unique city for Internet transmissions, as it’s geographic location gives nodes access to all of Asia, as well as the rest of the United States. One point to note is that Seattle is not a source node for IP2Location, meaning that all traffic through the city is transmission traffic, with no source traffic like the previous two cities discussed. Seattle is generally the main receiver of Asian Internet traffic and routes a majority of the traffic between Asia and Western Europe giving it responsibility for and access to the world’s most heavily populated countries. Seattle can be described as a no source traffic, extremely high transmission traffic city.
\begin{figure}[h!]
  \caption{Hits in Each North American City}
  \centering
    \includegraphics[width=0.5\textwidth]{CityHits.png}
\end{figure}
\subsection{Firewalls}

We translated the traceroute data into Google Charts (Geo-charts that allow us to represent data on a world map). Below, we graphed the number of hits from the traceroute data for each country and put it on a world map (Figure 4). It is apparent that the traffic in each country is what we expected, with the U.S., Canada, and much of Western Europe all green to represent that each of these countries had a large number of hits during the traceroute collection. One country that produced confusing results was China. China aggregated a total of 1,238 hits in the entire country – a number on par with Slovakia’s 1,268 hits. Taking those numbers into account, the comparison of the number of universities in each country really challenged our data. China with 389 universities (via univ.cc), to Slovakia’s 33, China has more than ten times the number of universities, and had ten times the number of destination addresses in our traceroute queries. China should receive approximately ten times the number of hits that was reflected in our data. However, there must be a reasonable explanation for these lower-than-expected numbers. One possible explanation is that our traceroute packets randomly got dropped. However, this is highly unlikely as China should have many more hits; random dropped packets probably could not cause this massive discrepancy. Perhaps the ‘Golden Shield Project’ of China, a censorship program in China to block unfavorable international packets from entering the country, is the reason for China’s unusually low number of hits. Obviously not all packets to China destinations were dropped, but the numbers are certainly skewed. 

\begin{figure}[h!]
  \caption{Hits Per Country}
  \centering
    \includegraphics[width=0.5\textwidth]{CountryHitsYellow.png}
\end{figure}

Just as Cheswick and Burch's Traceoute \cite{ches} noticed nodes disappearing and reappearing on the Yugoslavian network due to power grid outages during the 1999 Serbian war, similar results were seen in this research due to erection of firewalls. We ran the IP2Location Traceroute successfully to IP address 46.36.195.3 located in Antarctica in February 2015 as a test case. We did not anticipate having to save the route on our laptop, as we expected to run the route successfully again with the same IP address in May 2015 at the time of this writing. The successful February Traceroute execution showed the packet reaching Antarctica via the router on the RIPE network in Oxford. However, running the IP2Location Traceroute in May 2015 to any of the RIPE network runs into a firewall showing only the hops before leaving the starting servers network. Starting at any of the 12 IP2Location servers resulted in the final hop being close to the starting hop. Therefore, the intelligent VisualRoute tool was used to inspect possible firewall cases. Tracing to the "ripe.net" URL, VisualRoute showed a firewall icon after all the hops leading up to the point on the route where ICMP responses were not received. Similar behaviors are encountered when trying to route to North Korea's kcna.kp with IP address 175.45.177.74. The VisualRoute execution would show responses from en route routers until peering with the Chinese CNC Group. After which, there would be "No responses for this section of the route."


Routing to some IP addresses including 195.66.225.2 from the United States or Canada resulted in reaching a firewall at the destination. However, routing from Oslo and Paris resulted in a successful, complete Traceroute to the IP address. Therefore, firewalls were found targetted towards particular sources.

Similar results are seen when trying to route to nodes into private LANs from outside the LAN. For example, routing within a private LAN,
there is a router with IP address 10.170.200.152 However, this IP address is not reachable by IP2Location. VisualRoute, when using the HMA Pro VPN, shows a firewall on the entire route indicating lack of reachability.
\subsection{Bidirectional Routing}

Using the HMA Pro VPN and VisualRoute, we chose to trace a route from a known IP address in Helsinki, Finland to the known IP address of the IP2Location server in Dallas. From the IP2Location server in Dallas, we traced a route to the IP address in Helsinki. Both routes had SoftLayer Technologies as its transit core network. This result showed that routes between a source location and a destination can be bidirectional under some circumstances. 
To show that bidirectional routes are not always taken, we performed the same procedure by setting the VPN to Jacksonville, Florida. The route from Dallas to Jacksonville transited through the SoftLayer and Altrato networks. However, the reverse route transited through Zayo. Depending on the network topologies, a different routes may be taken based on the direction of the connection. Distance is not the deciding factor in routing as routing between the French cities of Nanterre and Paris involved transit via the United States through the Cogent Communications network.
\subsection{BGP Interdomain Routing}

BGPlay did correspond with our VisualRoute data with the HMA Pro VPN set to Reyjkavik, Iceland, and Renqiu, China. The routes were run to IP address 193.160.39.1. This IP address' ISP name as checked on both BGPlay and IP2Location was "Mr Dan Luedtke" with AS Number 57821. Routing to Mr. Leudtke from Iceland required a final peer with Level3 (AS 3365) to Mr. Leudtke. Routing from China resulted in a final peer with GTT (AS 12586) and Mr. Leudtke. Both were shown on the BGPlay graph as links from AS 57821. BGPlay was a good tool in understanding interdomain routing, but for situations where one or two ISPs were involved, analyzing BGP was not an effective approach.

\subsection{Physical Media}


All of the most common ISPs visited in our Traceroute results were transit cores using fiber optics to connect their networks. Our top ISP, Cogent Communications claims 59,000 miles of long-haul fiber with 27,000 miles of metropolitan fiber stretching between Asia, North America, and Europe with 610 metropolitan rings. Cogent employs IPv6 and multiprotocol label switching (MPLS) to provide 58,300 Gbps networking capacity to connect 5,270 ASes.  The end-to-end optical transport consists of IP-over-WDM providing up to 580 Gbps intercity capacity and 320 Gbps on metropolitan rings [citation]. Zayo, another transit core ISP from our Traceroute data, is a leader in the dark-lit fiber industry. Dark-lit fiber meets the demand for low-latency transport between metropolitan networks, especially between financial centers. Dark fiber also provides for wireless backhaul between LTE cellular towers and has about 6000 times the data rate of a T1 line [citation needed] where a T1 line carries 1.544 Mbps. 

Some end hosts such as desktops, laptops, and printers connect to the Internet by some form of Ethernet (IEEE 802.3) or Wi-Fi (IEEE 802.11). Other devices including smartphones and tablets can additionally connect to cellular networks to reach the Internet anywhere in the city. At this point in time, we assumed Wi-Fi and Ethernet were available in every country, so we did not focus our studies there. Additionally, Wi-Fi and Ethernet have limited range, typically to a single building, so we became most interested in cellular networks which provide connectivity anywhere in the city to their subscribers. We inspected the tables on Wikipedia of mobile network operators,  their available technologies, and their estimated population and penetration of subscribers. While we preferred a more reputable data source, no similarly comprehensive list of data was found for use in this research that included cellulrar statistics for all the countries on all of the continents. We noticed that cellular technology is highly influenced by the market. Most of the world has reached third generation (3G) technologies of Universal Mobile Telecommunications System (UMTS) with High-Speed Downlink Packet Access (HSDPA). More cellular countries have Evolved High-Speed Packet Access (DC-HSPA+).  HSDPA is based on the Global System for Mobile Communications (GSM) standard while DC-HSPA+ is based on Wideband Code Division Multiple Access (WCDMA). Fourth generation technologies are Worldwide Interoperability for Microwave Access (WiMAX IEEE 802.16) and Long Term Evolution (LTE). The lowest level of service seen is GSM-900 and General packet radio service (GPRS). In the Americas a form of 3G technology is Evolution Data-Optimized (EV-DO), an extension of the CDMA2000 standard.
The poorest continental nations in addition to several oceanic island nations in terms of mobile communications are Iraq, Yemen, and Eritrea as of 2014.These nations do not yet have data access, but they are at GSM-900, so text messages (SMS) and voice calls are still an option.  

We noticed a trend between subscriber population and the cellular technologies available from the cellular carriers in several countries. As seen in the table at right, VivaCell was offering up to LTE service. However, Beeline and Orange with about half a million Armenian subscribers offer up to HSDPA and DC-HSPA+, respectively. Therefore, the most popular carriers are more likely to have 4G technologies. The technologies seem to be most correlated with the extrinsic number of subscribers rather than the intrinsic penetration rate. For example, Albania had a 125 percent active user penetration rate as of January 2014, but mostly all the carriers were offering HSPA+. However, Belgium had a 114 percent penetration rate in 2009, but all the carriers are offering up to 1800 MHz LTE. The extrinsic number of subscribers is the determining factor. Mobistar Belgium, the least popular carrier in Belgium, had 3.375 million subscribers while Vodafone Albania, the most popular carrier in Albania had 1.659 million subscribers. Therefore, carriers with more millions of subscribers generally offer their country of operation better communication technologies than the carriers with fewer subscribers. This property applies to all continents, not just Europe. In Zimbabwe, a country in Africa, Econet offers up to 1800 MHz LTE and WiMAX to its 5.7 million subscribers (February 2012 statistic) whereas Telecel offers only up to HSDPA. In Americas including the Caribbean and Atlantic islands almost all carriers are offering LTE service. However, the more popular carriers in the Americas offer LTE at higher frequency bands to support higher data rates. 
        
The United States in-flight ISP, GoGo, uses Air-To-Ground technology to provide Internet services more than ten thousand feet from the ground. The Air-To-Ground technology is an extension of cell towers connecting cellular devices on the ground. Flying over the continental United States, the routers on GoGo-enabled passenger airplanes connect to the best signal found from the nearest mobile broadband cell towers below. The Air-To-Ground technology is unique to the GoGo ISP in the continental United States. 

However, in-flight connectivity is available internationally in the air and over the sea via Inmarsat satellites. GoGo, OnAir, and AeroMobile connect to satellites to provide international in-flight Internet connectivity in the air. However, the satellite signals are insufficient over the poles for connectivity, and video streaming is still unsupported. At this time, in-the-sky broadband connectivity is only available over the continental United States with GoGo. Over the seas, cruise lines such as Carnival use Inmarsat Internet satellites as well for their on-board WiFi LANs.

With local WiFi and Ethernet, cellular, fiber, Air-To-Ground, and satellite physical technologies, the Internet is able connect the world together at different data rates. At this point in time, the physical technology is invented for anybody anywhere within Earth’s atmosphere to connect to the Internet, but there is room for growth to equalize every nation in terms of communication capabilities and data transmission rates. However, most of the world now has access to mobile connectivity in addition to in-home connectivity.
\\\\
Armenia (June 2014) \\
3.3 million subscribers | 120 percent penetration \\\\
\begin{tabular}{ |l|l|l| }
  \hline
  Rank & Operator & Technology\\ \hline
  1 & VivaCell-MTS & GSM-900/1800 (GPRS, EDGE) \\
  Subs: & 2.1 million & 2100 MHz UMTS, DC-HSPA+ \\
  &  & 2600 MHz LTE \\ \hline
  2 & Beeline & GSM-900/1800 (GPRS, EDGE) \\
  Subs: & 0.7 million & 2100 MHz UMTS, HSDPA \\ \hline
  3 & Orange & GSM-900/1800 (GPRS, EDGE) \\
  Subs: & 0.5 million & 900/2100 MHz UMTS, DC-HSPA+ \\ \hline
\end{tabular}


\section{Conclusions and Future Work}
The results presented in this paper give a small visualization of the Internet routers world-wide, and pose interesting questions regarding the makeup of Internet topology. The results and analysis is simply our interpretation of the collected data, and there are always more graphs and questions to answer. The main take away from our research is that the graphical representations of data shown in this paper are largely influenced by the ISPs that control routing algorithms and BGP tables. ISPs route traffic for a number of reasons, such as in order to reach peering points, or route traffic across the globe as fast as possible. The baseline explanation for all Internet traffic is that the people, both in the United States and internationally, dictate what the ISPs do and offer. We see the countries with the highest population tend to have more universities, and countries with stricter governments have firewalls to limit our research.

For future research, it would be beneficial to normalize all data after collection. This would negate the bias seen for cities with source servers. Our main goal was to visualize the worldwide Internet traffic, and while we did a good job on collecting and mapping all data, some inconsistencies skew our maps. Our maps give a good analysis of cities with a strong ISP presence, and how traffic flows internationally. 


\bibliographystyle{ieeetr}
\bibliography{example}

\end{document}
