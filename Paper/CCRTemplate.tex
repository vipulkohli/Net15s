

%
% This file is an example file demonstrating how to use the sigcomm-alternate class for papers
% submitted in partial fulfillment for assignments in the Computer Networking and Distributed Systems
% classes, CSE 4344 and CSE 5344/7344, at Southern Methodist University. The sigcomm-alternate 
% LaTeX class is used for submissions to CCR. It is a goal of at least CSE 5344/7344 that the students
% will submit at least paper submitted for an assignment to CCR for publication; therefore, this example
% file conforms to the CCR submission guidelines.
%
% Author: Daniel W. Engels
% Creation Date: 17 January 2014
% Version: 0.1
%

\documentclass{sigcomm-alternate}

%---- The front matter contains packages being used, definitions, etc. including the title and author
%---- information for this work.

%---- Title
% The title of this work. For course assignments, the title must include the course number. 
\title{
Title of Work
}

%---- Authors
% The authors are placed all within a single LaTeX author box. This is a CCR guideline that is not true of
% all ACM publications.
\numberofauthors{4}

\author{
% Each author is identified by name, simple affiliation, and email address. The command \alignauthor (no
% curly braces required) must precede each author name. The command \affaddr (curly braces required) 
% must precede each line in the affiliation. The command \email{ } (curly braces required) must by used to 
% identify the email address for each author. When more than three authors exist, a tabular environment will
% need to be used to align the authors across multiple lines.
\alignauthor Izabel M. Great\\
\affaddr{Computer Science and Engineering Department\\
 	Southern Methodist University\\
	Dallas, Texas USA}\\
\email{imgreat@smu.edu}
%
\alignauthor Ignatio P. Freely\\
\affaddr{Computer Science and Engineering Department\\
 	Southern Methodist University\\
	Dallas, Texas USA}\\
\email{ipfreely@smu.edu}
%
\alignauthor Daniel W. Engels\\
\affaddr{Computer Science and Engineering Department\\
 	Southern Methodist University\\
	Dallas, Texas USA}\\
\email{dwe@smu.edu}
%\and  % use '\and' if you need 'another row' of author names
%------ Uncomment the following for editorial submissions to CCR
%\begin{tabular}{c}
%{\normalsize This article is an editorial note submitted to CCR. It has NOT been peer reviewed.}\\
%{\normalsize The authors take full responsibility for this article's
%technical content. Comments can be posted through CCR Online.}
%\end{tabular}
%}
}

%--- Begin the Document
\begin{document}

%---- Create the title
\maketitle

%---- Abstract
% The abstract is a 100 word summary of the problem addressed in the paper, the basic thesis and/or 
% the contributions of the paper, and the primary results and conclusions presented in the paper.
\begin{abstract}
The abstract summarizes the problem, the basic approach, and the primary results and conclusions of the work. This is effectively a 100 word ``paper trailer'' that captures the reader's attention and motivates them to read the entire paper even though the abstract gives away the plot and ending.
\end{abstract}

%---- Categories and Subject Descriptors
% Use this field to indicate the research categories within which this work is located. The \category command
% has three required fields and a fourth optional field.
\category{H.4}{Information Systems Applications}{Miscellaneous}

%---- General Terms
%
\terms{Theory}

%---- Keywords
%
\keywords{\LaTeX, CCR submission}

%----Main Body of the Paper

\section{Introduction}

\section{Conclusions}

\bibliographystyle{abbrv}
\bibliography{sigproc}  % sigproc.bib is the name of the Bibliography file located in the same directory as this file

\balancecolumns
%---- End the Document
\end{document}




